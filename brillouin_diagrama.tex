\newcommand\xsla{-1.2}
\newcommand\ysla{0.505}
\newcommand\hexheight{3em}
\newcommand\hexside{2em}

\newcommand\hex[3][]{
  \begin{scope}[
      #1,
      xscale=-1,
      yshift=#3,
      yslant=\ysla,
      xslant=\xsla,
      every node/.style={anchor=west,regular polygon,regular polygon sides=6,draw,inner sep=\hexside},
      transform shape
    ]
    \node (hex_#2) {};
  \end{scope}
}

\newcommand\hexhidden[3][]{
  \begin{scope}[
      #1,
      xscale=-1,
      yshift=#3,
      yslant=\ysla,
      xslant=\xsla,
      every node/.style={anchor=west,regular polygon,regular polygon sides=6,inner sep=\hexside,draw,dotted},
      transform shape
    ]
    \node (hex_#2) {};
  \end{scope}
}

\begin{figure}[b]
  \centering
  \begin{tikzpicture}
    \hex{top}{\hexheight}
    \hexhidden{middle}{0}
    \hex{bottom}{-\hexheight}

    \foreach \corn in {1,...,6}
    \draw (hex_top.corner \corn) -- (hex_bottom.corner \corn);

    % Caminho no qual as bandas são calculadas
    \draw[thick,myarrow=0.5] (hex_middle.center) -- (hex_middle.corner 4);
    \draw[thick,myarrow=0.5] (hex_middle.corner 4) -- (hex_middle.side 3);

    % Pontos de simetria
    \draw[fill=black] (hex_middle.center) circle (2pt) node[right]{$\Gamma$};
    \draw[fill=black] (hex_middle.corner 4) circle (2pt) node[below left]{$K$};
    \draw[fill=black] (hex_middle.side 3) circle (2pt) node[below right]{$M$};
    \draw[fill=black] (hex_middle.corner 3) circle (2pt) node[above right]{$K'$};
    \draw[fill=black] (hex_top.center) circle (2pt) node[above right]{$A$};
  \end{tikzpicture}
  \caption{
    Zona de Brillouin para monocamadas de TMDCs e principais pontos de simetria.
    Geralmente as bandas de valência e de condução são calculadas ao longo desses
    pontos, como no caminho exemplificado.
  }
  \label{fig:brillouin}
\end{figure}